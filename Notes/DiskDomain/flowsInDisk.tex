\documentclass[]{article}

\usepackage[margin=1.0in]{geometry}
\usepackage{amsmath}
\usepackage{amsfonts}
\usepackage{amsthm}
\usepackage{graphicx}
\usepackage{amssymb}

\usepackage{mathtools}

%opening
\title{A Geodesic/Horocycle Flow Approach to Hejhal's Algorithm}
\author{Alex Karlovitz}
\date{}

\begin{document}
	
	\maketitle
	
\section*{Flows in the Disk Model}

We wish to describe the geodesic and horocycle flows in the disk model.
We do so by recalling the salient facts in the upper half plane model, all the while noting how the Cayley transform moves the behavior into the disk.

\subsection*{Matrix Groups Acting on Hyperbolic Space}

The orientation-preserving isometries on the upper half plane $\mathbb{H}$ are exactly those M\"obius transformations defined by PSL$(2, \mathbb{R})$.
Recall that we can map from $\mathbb{H}$ to the unit disk $\mathbb{D}$ via the Cayley transform
$$
C(z) = \frac{z - i}{z + i}
$$
We can of course map back by the inverse
$$
C^{-1}(z) = i\frac{1 + z}{1 - z}
$$
So for $g \in \text{PSL}(2, \mathbb{R})$ acting on $\mathbb{H}$, the equivalent action on $\mathbb{D}$ is $CgC^{-1}$.
One can show that
$$
C\text{PSL}(2, \mathbb{R})C^{-1} = \text{PSU}(1, 1) :=
\left\{ 
\begin{pmatrix}
\alpha & \beta \\
\bar{\beta} & \bar{\alpha}
\end{pmatrix} \in M(2, \mathbb{C}) : |\alpha|^2 - |\beta|^2 = 1
\right\} / \{\pm I\}
$$
where we are thinking of $C$ as the matrix $\begin{psmallmatrix} 1 & -i \\ 1 & i \end{psmallmatrix} \in \text{PSL}(2, \mathbb{C})$, i.e., the usual identification of M\"obius transformations with matrices.
\\

Next, recall that we extend the action of PSL$(2, \mathbb{R})$ on $\mathbb{H}$ to one on $T^1\mathbb{H}$ via
$$
Dg(z, v) = \left( \frac{az + b}{cz + d}, \frac{1}{(cz + d)^2}v \right)
$$
where $g = \begin{psmallmatrix} a & b \\ c & d \end{psmallmatrix}$.
One can check that
\begin{enumerate}
	\item $D$ is indeed an action
	\item $D$ preserves the Riemannian metric on $\mathbb{H} \times \mathbb{C}$
	\item PSL$(2, \mathbb{R})$ acts simply transitively on $T^1\mathbb{H}$
\end{enumerate}
Thus, we can identify PSL$(2, \mathbb{R})$ with $T^1\mathbb{H}$ via this action by choosing a reference point in $T^1\mathbb{H}$.
The typical choice is $(i, i)$, so we identify $(z, v) \in T^1\mathbb{H}$ with the unique $g \in \text{PSL}(2, \mathbb{R})$ such that $Dg(i, i) = (z, v)$.
\\

To think about the corresponding extended action on $T^1\mathbb{D}$, we first need to extend the map $C: \mathbb{H} \rightarrow \mathbb{D}$ to a map from $T^1\mathbb{H}$ to $T^1\mathbb{D}$.
The most natural such extension is
$$
DC(z, v) = (C(z), C'(z)v) = \left( \frac{z - i}{z + i}, \frac{2i}{(z + i)^2}v \right)
$$
(see Appendix A for why this extension is natural).
One easily verifies that if $(z, v) \in T^1\mathbb{H}$ then $DC(z, v) \in T^1\mathbb{D}$.
(Recall that the Riemannian metric on the tangent space at the point $p \in \mathbb{D}$ is given by $\langle v, w \rangle_p = 4v\cdot w/(1 - |p|^2)^2$).
\\

Now that we can map between $T^1\mathbb{H}$ and $T^1\mathbb{D}$, we can extend the action of $\text{PSU}(1, 1)$ on $\mathbb{D}$ to one on $T^1\mathbb{D}$.
By passing through $T^1\mathbb{H}$ on the way, we see that the action should be $(DC)(Dg)(DC)^{-1}$ where $\gamma = CgC^{-1}$ is an element of $\text{PSU}(1, 1)$.
It is straightforward to check that this is equivalent to the action
$$
D\gamma(z, v) = (\gamma(z), \gamma'(z)v) = \left( \frac{\alpha z + \beta}{\bar{\beta}z + \bar{\alpha}}, \frac{1}{(\bar{\beta}z + \bar{\alpha})^2}v \right)
$$
where
$$
\gamma =
\begin{pmatrix}
\alpha & \beta \\
\bar{\beta} & \bar{\alpha}
\end{pmatrix} \in \text{PSU}(1, 1)
$$
In other words, the matrix action is the same as the one we are used to in the upper half plane model; it is just a different matrix group for the disk.
\\

Finally, just as we identified $\text{PSL}(2, \mathbb{R})$ with $T^1\mathbb{H}$, we can identify $\text{PSU}(1, 1)$ with $T^1\mathbb{D}$.
We will choose the reference point $(1, 1/2) \in T^1\mathbb{D}$ (since this is the image of the old reference point $(i, i)$ under $DC$).
That is, $\text{PSU}(1, 1)$ is identified with $T^1\mathbb{D}$ via
$$
\gamma \longleftrightarrow D\gamma\left( 1, \frac{1}{2} \right)
$$

\subsection*{Geodesic Flow}

Recall that the geodesics in $\mathbb{H}$ are vertical lines and semicircles perpendicular to the real line (in other words, the top half of circles with centers on the real axis).
One can show that a point in $\mathbb{H}$ paired with a direction uniquely defines a geodesic.
In other words, each $(z, v) \in T^1\mathbb{H}$ gives a geodesic.
Of course, different points in the tangent bundle can lead to the same geodesic; we typically think of this as taking different \textit{starting points} on the same geodesic.
\\

The \textit{geodesic flow} beginning at a point $(z, v) \in T^1\mathbb{H}$ is defined by flowing along the geodesic defined by $(z, v)$ in the direction of $v$ at a constant speed (speed is defined with respect to the Riemannian metric).
Thus, for each starting point $g \sim (z, v) \in T^1\mathbb{H}$ (where we write $g$ for the matrix in $\text{PSL}(2, \mathbb{R})$ corresponding to $(z, v)$), geodesic flow $g_t$ is a function of time.
Note that $g_0 = (z, v)$ and $g_t$ for $t < 0$ refers to flowing in the opposite direction (i.e., $-v$).
\\

We can describe geodesic flow very easily in terms of matrices.
First, one computes that the geodesic flow starting at $(i, i)$ is given by $(e^ti, e^ti)$.
In terms of $\text{PSL}(2, \mathbb{R})$ - where $(i, i)$ is identified with $I$ - the flow is given by the matrix
$$
a_t =
\begin{pmatrix}
e^{t/2} & ~ \\
~ & e^{-t/2}
\end{pmatrix}
$$
(This is easily checked by verifying that $a_t \sim (e^ti, e^ti)$ in the identification).
Next, we can observe that the geodesic flow starting at any point $(z, v)$ is equivalent to mapping the flow starting at $(i, i)$ by the matrix $g \in \text{PSL}(2, \mathbb{R})$ which has $g \sim (z, v)$.
This works because $Dg$ is an isometry.
Therefore, geodesic flow for time $t$ starting at $g \sim (z, v)$ is succinctly described as
$$
g_t = ga_t
$$

Next, we would like to describe the geodesic flow in the disk model.
Recall that the geodesics in the disk are diameters plus half circles normal to the boundary.
We can use the Cayley transform as before to map between the models.
The element corresponding to $a_t$ is
$$
c_t := Ca_tC^{-1} =
\begin{pmatrix}
\cosh\frac{t}{2} & \sinh\frac{t}{2} \\
\sinh\frac{t}{2} & \cosh\frac{t}{2}
\end{pmatrix}
$$
The action of this on $T^1\mathbb{D}$ is thus
$$
Dc_t(z, v) = \left( \frac{\left( \cosh\frac{t}{2} \right)z + \sinh\frac{t}{2}}{\left( \sinh\frac{t}{2} \right)z + \cosh\frac{t}{2}}, \frac{1}{\left( \left( \sinh\frac{t}{2} \right)z + \cosh\frac{t}{2} \right)^2}v \right)
$$
A quick check that this action makes sense is to note that $Dc_t(1, 1/2)$ does indeed give the interval $(-1, 1)$ on the real line.

\section*{Appendix A: Extension of $C$ to the Tangent Bundles}

One way to find an appropriate extension is to consider the unique geodesic going through any point $(z, v) \in T^1\mathbb{H}$.
Recall that the geodesic passing through that point is the set $\{ga_t(i): t\in \mathbb{R}\}$ where $g$ is the unique element of $\text{PSL}(2, \mathbb{R})$ with $g(i, i) = (z, v)$.
So the corresponding geodesic in $\mathbb{D}$ is given by $C(ga_ti)$.
Taking the derivative with respect to $t$, we see that the tangent vectors are given by
$$
C'(ge^ti)g'(e^ti)e^ti
$$
We are interested in the point in $T^1\mathbb{D}$ corresponding to $(z, v)$ so we simply take $t = 0$ and see that the tangent vector should be
$$
C'(z)v
$$

\end{document}