\documentclass[]{article}

\usepackage[margin=1.0in]{geometry}
\usepackage{amsmath}
\usepackage{amsfonts}
\usepackage{amsthm}
\usepackage{graphicx}
\usepackage{amssymb}

\usepackage{mathtools}

\newtheorem*{theorem}{Theorem}

%opening
\title{Asymptotics for the Fourier Coefficients of Maass Forms in Various Expansions}
\author{Alex Karlovitz}
\date{}

\begin{document}
	
	\maketitle

Let $\Gamma$ be a Fuchsian group, and let $f$ be a Maass form on $\Gamma\backslash\mathbb{H}$.
We assume that $f$ has some type of Fourier expansion (e.g. at a cusp, at a flare, or in the polar expansion in the disk model).
In particular, we assume $f$ has some expression of the form
$$
	f(x, y) = \sum_{n \in \mathbb{Z}}a_nW(y)e(nx)
$$
(Depending on the context, we may instead have a logarithmic Fourier expansion, or an expansion in coordinates other than $(x, y)$).

The purpose of this note is to derive a bound for the $a_n$'s as $|n| \rightarrow \infty$.
In particular, we are planning on cutting the Fourier expansion off at a finite point $|n| \leq M$, so we wish to know for a given $y$ what is a good choice for $M$.
Before doing this in the flare expansion, we recall an argument for the typical cuspidal expansion.
Then, we attempt to replicate that argument in the flare case.
	
\section*{Asymptotics in the Cuspidal Expansion}

When the group $\Gamma$ in question has a cusp, recall that a Maass form $f(z)$ has a Fourier expansion in the $x$-variable of the form
\begin{equation}\label{cuspExp}
f(x + iy) = \sum_{n\in\mathbb{Z}}a_n\sqrt{y}K_\nu(2\pi|n|y)e(nx)
\end{equation}
It can be shown that $K_\nu(Y)$ decays exponentially as $Y \rightarrow \infty$.
On the other hand, note that
$$
a_n\sqrt{y}K_\nu(2\pi|n|y) = \int_{0}^{1}f(x + iy)e(-nx)dx
$$
We will use the fact that $f \in L^2(\Gamma\backslash\mathbb{H})$ to get a bound on the $a_n$'s.
For simplicity, let us assume for the remainder of the argument that $n > 0$ (the $n < 0$ case follows similarly).
First, we multiply both sides of the above equation by $1/y^2$ and integrate over $y$ in $1/n$ to $2/n$ (assuming of course that $n \neq 0$).
This bounds $K_{\nu}(2\pi ny)$ between two constants, and hence we conclude that
$$
\left|a_n\int_{1/n}^{2/n}\sqrt{y}\frac{dy}{y^2}\right| \leq C_0\left|\int_{1/n}^{2/n}\int_{0}^{1}f(x+iy)e(-nx)\frac{dxdy}{y^2}\right|
$$
where $C_0 > 0$ is a constant independent of $n$.
The integral on the left hand side comes out to a positive constant times $\sqrt{n}$, and so we conclude that
\begin{equation}\label{firstCuspBound}
	|a_n| \leq
	Cn^{-1/2}\int_{1/n}^{2/n}\int_{0}^{1}|f(x+iy)|\frac{dxdy}{y^2}
\end{equation}
where $C > 0$ is a constant independent of $n$.
\\

The next part of the argument is to use equidistribution of low-lying horocycles to bound the inner integral (with respect to $x$) on the right hand side of (\ref{firstCuspBound}).

We use the statement of Theorem 1.11 in ``Almost Prime Pythagorean Triples in Thin Orbits'' by Kontorovich-Oh.
We begin by setting up some notation.
By Patterson and Lax-Philips (see references in AK-Oh's paper), the Laplacian acting on $L^2(\Gamma\backslash\mathbb{H})$ admits finitely many eigenvalues in $[0, 1/4)$.
We denote these eigenvalues
$$
0 \leq \lambda_0 < \lambda_1 < \cdots < \lambda_k < \frac{1}{4}
$$
and we write $\phi_j$ for the eigenfunction corresponding to $\lambda_j$.
Let us assume the eigenfunctions are scaled so that $||\phi_j||_2 = 1$.

Writing $\lambda_j = s_j(1 - s_j)$, where $s_j$ is chosen to be greater than $1/2$, one version of the theorem can be stated as follows:
\begin{theorem}
	Let $\psi \in C^\infty(\Gamma\backslash\mathbb{H})$. Then
	$$
	\int_{0}^{1}\psi(x + iy)dx =
	C\langle \psi, \phi_0 \rangle y^{1-\delta} + O\left( y^{1 - s_1} \right)
	$$
	as $y \rightarrow 0$, where $C > 0$ is independent of $\psi$, and the implied constant depends on a Sobolev norm $\psi$.
\end{theorem}

Next, we apply the theorem to the inner integral on the right hand side of \ref{firstCuspBound}.
Taking $\psi = |f|$, this implies that
$$
\int_{0}^{1}|f(x + iy)|dx =
C\langle |f|, \phi_0 \rangle y^{1-\delta} + O\left( y^{1 - s_1} \right)
$$
By Cauchy-Schwartz, the inner product is bounded by the $L_2$ norm of $f$.
Since $s_1 < \delta$, the previous equation implies that
\begin{equation}\label{xIntBound}
	\int_{0}^{1}|f(x + iy)|dx \ll_f y^{1 - \delta}
\end{equation}
\\

Putting together the bounds (\ref{firstCuspBound}) and (\ref{xIntBound}), we get
$$
|a_n| \ll_f n^{-1/2}\int_{1/n}^{2/n}y^{-1-\delta}dy
$$
That is,
\begin{equation}\label{cuspBound}
	|a_n| \ll_f n^{\delta-1/2}
\end{equation}
(Recall that $1/2 < \delta < 1$).
\\

From the arguments above, we see that the entire Fourier coefficient
$$
a_n\sqrt{y}K_\nu(2\pi|n|y)
$$
decays exponentially with $n$.
In particular, beyond $n \approx 1/y$ the coefficients will decay very quickly.
So for a given $y$, we can say that the finite expansion
$$
\sum_{|n| \leq M}a_n\sqrt{y}K_\nu(2\pi|n|y)e(nx)
$$
is a very good approximation to the function if we take, say, $M \approx 2/y$.

Let us take the preceding argument one step further.
Since the function $2/y$ is decreasing in $y$, we can make a uniform statement in any region where $y$ is bounded below.
For example, we can conclude that taking $M = 20$ will result in a very accurate approximation for every $y \geq 1/10$.
\\

\textbf{Note:} if we are only interested in the case of where $f$ is the base eigenfunction of the Laplacian, we can use positivity to make the argument much simpler.
This is the argument we use in the disk model below.

\section*{Asymptotics in the Flare Expansion}

When the group $\Gamma$ in question has a flare, recall that a Maass form $\phi(z)$ has a Fourier expansion in the $r$-variable (where we write $z = (r, \theta)$ in polar coordinates).
The expansion depends on the size of the flare, which we assume to bounded between $1$ and $\kappa > 1$.
The expansion then takes the form
\begin{equation}\label{flareExp}
\phi(r, \theta) = \sum_{n\in\mathbb{Z}}b_n\sqrt{\sin\theta}P_{\mu_n}^{-\nu}(\cos\theta)e\left( n\frac{\log r}{\log\kappa} \right)
\end{equation}
where
$$
\mu_n = -\frac{1}{2} + \frac{2\pi in}{\log\kappa}
$$

The asymptotics for the Fourier coefficients are shown in the appendix to the paper ``Almost Prime Pythagorean Triples in Thin Orbits'' by Kontorovich-Oh.
There, they show
$$
b_n \ll n^se^{-\pi n\alpha/\log\kappa}
$$
where
\begin{itemize}
	\item $\Delta\phi = s(1-s)\phi$
	\item $\alpha$ is chosen so that the flare
	$$
	\{ re^{i\theta} : 1 < r < \kappa, 0 < \theta < \alpha \}
	$$
	is entirely contained in a single fundamental domain for $\Gamma\backslash\mathbb{H}$
	\item the implied constant depends on $\alpha$ and $\kappa$
\end{itemize}
(Note: the statement above is ambiguous with regards to the choice of $s$ versus $1 - s$.
Perhaps check the proof to determine correct choice? In any case, the exponential behavior dominates the polynomial behavior, so it shouldn't affect the final outcome.)

As shown in the subsection below on the asymptotics of the Legendre $P$ function, we have
$$
P_{\mu_n}^{-\nu}(\cos\theta) \sim (\sin\theta)^{-1/2}|n|^{-\nu-1/2}e^{2\pi|n|\theta/\log\kappa}
$$
Thus the entire Fourier coefficient
$$
b_n\sqrt{\sin\theta}P_{\mu_n}^{-\nu}(\cos\theta)
$$
decays exponentially as $n \rightarrow \infty$ so long as $\theta < \frac{\alpha}{2}$.
\\

\textbf{Note:} experimental results suggest that the $b_n$'s in fact decay like
$$
b_n \sim e^{-2\pi^2n/\log\kappa}
$$
This is important if we want to try rescaling the coefficients for numerical stability.

\subsection*{Asymptotics of the Legendre $P$ Function}

We make use of the Digital Library of Mathematical Functions (DLMF) to get asymptotics on the Legendre $P$ function.
We start with equation 14.15(iii) in the DLMF:
$$
P_\mu^{-\nu}(\cos\theta) = \frac{1}{\mu^\nu}\sqrt{\frac{\theta}{\sin\theta}}\left( J_\nu((\mu+1/2)\theta) + O\left( \frac{1}{\mu} \right)\text{env}J_\nu((\mu+1/2)\theta) \right)
$$
where $J_\nu$ is the $J$-Bessel function and $\text{env}J_\nu$ is defined in the DLMF.
This statement holds uniformly for $\theta \in (0, \pi - \delta]$ as $|\mu| \rightarrow \infty$ for fixed $\nu \geq 0$.
Since we are interested in the behavior as $|\mu|$ gets large, we will content ourselves with the statement
$$
P_\mu^{-\nu}(\cos\theta) \sim \frac{1}{\mu^\nu}\sqrt{\frac{\theta}{\sin\theta}}J_\nu((\mu+1/2)\theta)
$$
as $|\mu| \rightarrow \infty$.

Next, we need asymptotics for the $J$-Bessel function.
Equation 10.7(ii) in the DLMF states that
$$
J_\nu(z) = \sqrt{\frac{2}{\pi z}}\left( \cos\left(z - \frac{\nu\pi}{2} - \frac{\pi}{4}\right) + e^{|\text{Im}(z)|}o(1) \right)
$$
for fixed $\nu$ as $|z| \rightarrow \infty$.
In particular, if $\text{Im}(z)$ is becoming large, we have that
$$
J_\nu(x + iy) \sim y^{-1/2}e^{|y|}
$$
as $y \rightarrow \pm\infty$.

Putting these estimates together, we expect for
$$
\mu_n := -\frac{1}{2} + \frac{2\pi in}{\log\kappa}
$$
to have
\begin{equation}\label{legPAsymptotic}
P_{\mu_n}^{-\nu}(\cos\theta) \sim (\sin\theta)^{-1/2}|n|^{-\nu-1/2}e^{2\pi|n|\theta/\log\kappa}
\end{equation}
as $|n| \rightarrow \infty$.
Note that this asymptotic holds \textit{uniformly for} $\theta \in (0, \pi - \delta]$.

\section*{Asymptotics in the Disk Model}

Recall that we can work in the disk model (as opposed to the upper half plane model) via the map $\tau : \mathbb{H} \rightarrow \mathbb{D}$ defined by
$$
\tau(z) = \frac{z - i}{z + i}
$$
If we write our Maass form in polar coordinates $f(\rho, \theta)$ on the disk, we automatically have a Fourier expansion in the disk model due to the $2\pi$-periodicity of $f$ in $\theta$.
Applying the differential equation $\Delta f + s(1-s)f = 0$, we get an explicit expression for the Fourier coefficients.
In this way, one can show that the expansion takes the form
$$
f(\rho, \theta) = \sum_{n \in \mathbb{Z}}c_n(1 - \rho^2)^s\rho^{|n|}\prescript{}{2}{F_1}(s, s + |n|, 1 + |n|, \rho^2)e^{in\theta}
$$
We can then solve for the $n^{th}$ Fourier coefficient:
$$
c_n(1 - \rho^2)^s\rho^{|n|}\prescript{}{2}{F_1}(s, s + |n|, 1 + |n|, \rho^2) =
\int_{0}^{2\pi}f(\rho, \theta)e^{-in\theta}d\theta
$$
Now, we are interested in the case where $f = \phi_0$ is the base eigenfunction.
In this case, $f$ is everywhere positive, so we have
\begin{equation}\label{firstCbound}
|c_nW_n(\rho)| \leq \int_{0}^{2\pi}|f(\rho,m \theta)|d\theta = c_0W_0(\rho)
\end{equation}
where we are writing $W_n(\rho)$ for the $n^{th}$ Whittaker function which appears in the Fourier expansion.
(Side note: inequality (\ref{firstCbound}) immediately implies $c_0 \neq 0$).
Next, we apply the asymptotic
$$
\prescript{}{2}{F_1}(s, s + |n|, 1 + |n|, \rho^2) \rightarrow (1 - z)^{-s}
$$
as $n \rightarrow \infty$. (Still need to prove this, but seems likely since $\prescript{}{2}{F_1}(a, b, b, z) = (1 - z)^{-a}$; I also checked this computationally in Python and the result appears to hold).
Since $0 \leq \rho < 1$, we see that the entire Whittaker function has
$$
W_n(\rho) = (1 - \rho^2)^s\rho^{|n|}\prescript{}{2}{F_1}(s, s + |n|, 1 + |n|, \rho^2) \sim \rho^{|n|}
$$
as $n \rightarrow \infty$.
Plugging this into (\ref{firstCbound}), we see that
$$
c_n \ll \rho^{-|n|}
$$
as $n \rightarrow \infty$.
Now, this was true for any $\rho \in [0, 1)$.
The idea now is to take $\rho = 1 - \delta$, with $\delta > 0$ small.
Then for any $\rho < 1 - \delta$, the entire Fourier coefficient has exponential decay:
$$
c_nW_n(\rho) \sim \left( \frac{\rho}{1 - \delta} \right)^{|n|}
$$
Thus, inside any fixed radius $\rho \leq P$, there is a constant $M$ so that taking the finite Fourier expansion $|n| \leq M$ gives a good approximation to the Maass form.

\section*{Old Ideas}

\subsection*{Attempts at the Flare Expansion before finding Reference}

There are two parts to the Fourier coefficient: the unknown $b_n$'s and the known function of $\theta$.
For the function of $\theta$, it can be shown that
$$
P_{\mu_n}^{-\nu}(\cos\theta) \sim (\sin\theta)^{-1/2}|n|^{-\nu-1/2}e^{2\pi|n|\theta/\log\kappa}
$$
as $|n| \rightarrow \infty$ uniformly in $\theta \in (0, \pi-\delta]$ for any $\delta > 0$.
For a proof, see the section below on asymptotics of the Legendre $P$ function.
On the other hand, note that
$$
b_n\sqrt{\sin\theta}P_{\mu_n}^{-\nu}(\cos\theta) =
\int_{1}^{\kappa}\phi(r, \theta)e\left(-n\frac{\log r}{\log\kappa}\right)\frac{dr}{r}
$$
We will use the fact that $\phi \in L^2(\Gamma\backslash\mathbb{H})$ to get a bound on the $b_n$'s.
First, we apply the Cauchy-Schwartz inequality:
$$
|b_n\sqrt{\sin\theta}P_{\mu_n}^{-\nu}(\cos\theta)|^2 =
\left|\int_{1}^{\kappa}\phi(r, \theta)e\left(-n\frac{\log r}{\log\kappa}\right)\frac{dr}{r}\right|^2 \leq
\int_{1}^{\kappa}|\phi(r, \theta)|^2\frac{dr}{r^2}
$$
Next, since this holds for all $\theta \in (0, \pi)$, we can multiply by $(\sin\theta)^{-2}$ and integrate over any $I \subseteq (0, \pi)$.
\\

\textbf{STILL NEED TO FIGURE OUT THE REST OF THIS ARGUMENT. HERE ARE MY IDEAS SO FAR.}

Continuing with our previous inequality, we see that
$$
|b_n|^2\int_I \left( P_{\mu_n}^{-\nu}(\cos\theta) \right)^2\frac{d\theta}{\sin\theta} \leq
\int_I\int_{1}^{\kappa}|\phi(r, \theta)|^2\frac{drd\theta}{r^2\sin^2\theta}
$$
Now the hyperbolic measure in polar coordinates is just
$$
\frac{dxdy}{y^2} = \frac{rdrd\theta}{r^2\sin^2\theta} = \frac{drd\theta}{r\sin^2\theta}
$$
So the right hand side in our inequality above is bounded by the $L^2$ norm of $\phi$ times the number of fundamental domains our domain of integration passes through.
In other words, we find that
$$
|b_n|^2\int_I \left( P_{\mu_n}^{-\nu}(\cos\theta) \right)^2\frac{d\theta}{\sin\theta} \leq
C\cdot\#\{ \mathcal{F} : \mathcal{F} \cap ([1, \kappa]\times I) \neq \varnothing \}
$$

\textbf{Thoughts:} smaller values of $\theta$ will ensure we pass through fewer fundamental domains.
For example, if $I = (0, \pi/2)$, then the size of the set on the right is just 1!
So for bounding the right hand side, small $\theta$ is good.

On the other hand, the integrand on the left hand side is growing like an exponential which depends on $\theta$.
The larger $\theta$ is, the larger the integrand.
So when thinking about the left hand side, larger values of $\theta$ will lead to better bounds on $|b_n|$.

The question is clearly then to optimally pick $I$...
	
\end{document}