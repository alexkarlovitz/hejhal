\documentclass[]{article}

\usepackage[margin=1.0in]{geometry}
\usepackage{amsmath}
\usepackage{amsfonts}
\usepackage{amsthm}
\usepackage{graphicx}
\usepackage{amssymb}

\usepackage{mathtools}

%opening
\title{Recent Results}
\author{Alex Karlovitz}
\date{}

\begin{document}
	
	\maketitle
	
In this document, I collect results from tests using the PARI code.

\section*{Tests for Hejhal's Algorithm}

The code assumes we are working on a Hecke triangle group generated by $z \mapsto z + 1$ and $z \mapsto -R^2/z$ for some $R > 0$.
Here are a few values of $R$ and corresponding $\nu$ values which are known to come from a true Maass form.
\begin{itemize}
	\item $R = 1$, $\nu = i9.5336952613\dots$
	\begin{itemize}
		\item this is the group $SL(2, \mathbb{Z})$
	\end{itemize}
	\item $R = 1/\sqrt{2}$, $\nu = i7.220872\dots$
	\begin{itemize}
		\item this is the congruence group $\Gamma_4$
	\end{itemize}
	\item $R = 1/\sqrt{2}$, $\nu = i11.317680\dots$
	\begin{itemize}
		\item this is also the congruence group $\Gamma_4$, just a different eigenvalue
	\end{itemize}
	\item $R = 7/20$, $\nu = 0.26705241700910205677150208864259506276668(7)\dots$
	\begin{itemize}
		\item note that this is an infinite volume fundamental domain, since $7/20 < 1/2$
		\item note that $\nu$ is real in this example
	\end{itemize}
\end{itemize}

\section*{July 2020}

\subsection*{Tests Using all Three Expansions on Hecke Groups}

Here, we collect results using the code in test\_all\_Hecke.pari.
Specifically, we are running the function example\_secant().
\\

For each test, we list the input values, the output $s = \nu + 1/2$, and the approximate error from the true value.
\\

\textbf{Precision:} until noted, using default(realprecision, 40) for all tests.
\\

First, we want to vary which values are going into the secant method.
Let's fix the following input.

Input:
$$
r = 0.35000000000000000000000000000000 $$$$
y_0 = 0.30000000000000000000000000000000 ~~~ M_1 = 10 $$$$
\alpha_0 = 2.2000000000000000000000000000000 ~~~ M_2 = 10 $$$$
\rho_0 = 3/4 ~~~ M_3 = 10 $$$$
N_1 = 20 ~~~ N_2 = 20 ~~~ N_3 = 20 $$$$
s_{start} = 0.76000000000000000000000000000000 $$$$
\delta_{start} = 0.0050000000000000000000000000000000
$$

Now we look at different outputs depending on our choices for the secant method.
\\

Test using first two coefficients from cusp and flare expansions; output:
$$
s_{guess} = 0.76705175510227623235618536383837 $$$$
E \approx 6.6190682582441531672480422196285 E-7
$$

Test using first two coefficients from cusp expansion, plus first from flare and disk expansions; output:
$$
s_{guess} = 0.76705218622576186081031266563593 $$$$
E \approx 2.3078334019596118942300666740463 E-7
$$

Test using first two coefficients from each of the three expansions; output:
$$
s_{guess} = 0.76705175510227623235618536383837 $$$$
E \approx 6.6190682582441531672480422196285 E-7
$$

Test using first coefficient from each of the three expansions; output:
$$
s_{guess} = 0.76705146318540003884924317275672 $$$$
E \approx 9.5382370201792225891588587480777 E-7
$$

\textbf{From now on, tests use the first two coefficients from the cusp expansion, and the first coefficient from the flare and disk expansions}.
As seen above, this is the method which performed best on the given example.
\\

Input:
$$
r = 0.3500000000000000000000000000000000000000 $$$$
y_0 = 0.3100000000000000000000000000000000000000 ~~~ M_1 = 20 $$$$
\alpha_0 = 2.300000000000000000000000000000000000000 ~~~ M_2 = 12 $$$$
\rho_0 = 3/4 ~~~ M_3 = 5 $$$$
N_1 = 22 ~~~ N_2 = 24 ~~~ N_3 = 6 $$$$
s_{start} = 0.7670524170000000000000000000000000000000 $$$$
\delta_{start} = 2.000000000000000000000000000000000000000 E-11
$$

Output:
$$
s_{guess} = 0.7670921794552911637659592720972066442132 $$$$
E \approx 3.976244618910699445718345461158144652910 E-5
$$

Input:
$$
r = 0.3500000000000000000000000000000000000000 $$$$
y_0 = 0.3100000000000000000000000000000000000000 ~~~ M_1 = 20 $$$$
\alpha_0 = 2.300000000000000000000000000000000000000 ~~~ M_2 = 12 $$$$
\rho_0 = 3/4 ~~~ M_3 = 5 $$$$
N_1 = 22 ~~~ N_2 = 24 ~~~ N_3 = 0 $$$$
s_{start} = 0.7670524170000000000000000000000000000000 $$$$
\delta_{start} = 2.000000000000000000000000000000000000000 E-11
$$

Output:
$$
s_{guess} = 0.7671124406019181695697260859671915065528 $$$$
E \approx 6.002359281611279822399732459644378613082 E-5
$$

Input:
$$
r = 0.3500000000000000000000000000000000000000 $$$$
y_0 = 0.3100000000000000000000000000000000000000 ~~~ M_1 = 20 $$$$
\alpha_0 = 2.300000000000000000000000000000000000000 ~~~ M_2 = 12 $$$$
\rho_0 = 3/4 ~~~ M_3 = 12 $$$$
N_1 = 22 ~~~ N_2 = 24 ~~~ N_3 = 0 $$$$
s_{start} = 0.7670524170000000000000000000000000000000 $$$$
\delta_{start} = 2.000000000000000000000000000000000000000 E-11
$$

Output:
$$
s_{guess} = 0.7670524170338548135342611762210143883645 $$$$
E \approx 2.475275676275908757841932559781421248042 E-11
$$

Input:
$$
r = 0.3500000000000000000000000000000000000000 $$$$
y_0 = 0.3100000000000000000000000000000000000000 ~~~ M_1 = 20 $$$$
\alpha_0 = 2.300000000000000000000000000000000000000 ~~~ M_2 = 12 $$$$
\rho_0 = 3/4 ~~~ M_3 = 20 $$$$
N_1 = 22 ~~~ N_2 = 24 ~~~ N_3 = 0 $$$$
s_{start} = 0.7670524170000000000000000000000000000000 $$$$
\delta_{start} = 2.000000000000000000000000000000000000000 E-11
$$

Output:
$$
s_{guess} = 0.7670524170091020512933774434170124639491 $$$$
E \approx 5.478124645225582598817609789207250709646 E-18
$$

Input:
$$
r = 0.3500000000000000000000000000000000000000 $$$$
y_0 = 0.3100000000000000000000000000000000000000 ~~~ M_1 = 20 $$$$
\alpha_0 = 2.300000000000000000000000000000000000000 ~~~ M_2 = 12 $$$$
\rho_0 = 3/4 ~~~ M_3 = 20 $$$$
N_1 = 22 ~~~ N_2 = 24 ~~~ N_3 = 6 $$$$
s_{start} = 0.7670524170000000000000000000000000000000 $$$$
\delta_{start} = 2.000000000000000000000000000000000000000 E-11
$$

Output:
$$
s_{guess} = 0.7670524170091018238937194898650120691583 $$$$
E \approx 2.328777825987775829936083804947119040820 E-16
$$

\textbf{Thoughts:} the above slew of tests suggests two things.
\begin{enumerate}
	\item Including a relatively high number of coefficients in the disk expansion ($M_3$) seems to be necessary.
	\item Taking test points from the disk model ($N_3$) seems to actually \textit{hurt} performance. Why are these points not useful (and even further, damaging)?
\end{enumerate}

Input:
$$
r = 0.3500000000000000000000000000000000000000 $$$$
y_0 = 0.3100000000000000000000000000000000000000 ~~~ M_1 = 20 $$$$
\alpha_0 = 2.300000000000000000000000000000000000000 ~~~ M_2 = 12 $$$$
\rho_0 = 3/4 ~~~ M_3 = 30 $$$$
N_1 = 22 ~~~ N_2 = 24 ~~~ N_3 = 0 $$$$
s_{start} = 0.7670524170000000000000000000000000000000 $$$$
\delta_{start} = 2.000000000000000000000000000000000000000 E-11
$$

Output:
$$
s_{guess} = 0.76705241700910205612288725661764384474 $$$$
E \approx 6.4861483202495121802575748064887076918 E-19
$$

Okay, so we have $s$ to about 19 decimal places. Let's start there and try to zoom in.
\\

Input:
$$
r = 0.3500000000000000000000000000000000000000 $$$$
y_0 = 0.3100000000000000000000000000000000000000 ~~~ M_1 = 26 $$$$
\alpha_0 = 2.300000000000000000000000000000000000000 ~~~ M_2 = 16 $$$$
\rho_0 = 3/4 ~~~ M_3 = 26 $$$$
N_1 = 28 ~~~ N_2 = 32 ~~~ N_3 = 0 $$$$
s_{start} = 0.7670524170091020567000000000000000000000 $$$$
\delta_{start} = 3.000000000000000000000000000000000000000 E-19
$$

Output:
$$
s_{guess} = 0.7670524170091020567 $$$$
E \approx 0.E-19
$$

\textbf{Note:} obviously some kind of coding issue came up in the previous test. Maybe if we increase precision to default(realprecision, 60)?
\\

\textbf{Precision:} until noted, using default(realprecision, 60) for all tests below.
\\

Input:
$$
r = 0.350000000000000000000000000000000000000000000000000000000000 $$$$
y_0 = 0.310000000000000000000000000000000000000000000000000000000000 ~~~ M_1 = 26 $$$$
\alpha_0 = 2.30000000000000000000000000000000000000000000000000000000000 ~~~ M_2 = 16 $$$$
\rho_0 = 3/4 ~~~ M_3 = 26 $$$$
N_1 = 28 ~~~ N_2 = 32 ~~~ N_3 = 0 $$$$
s_{start} = 0.767052417009102056700000000000000000000000000000000000000000 $$$$
\delta_{start} = 3.00000000000000000000000000000000000000000000000000000000000 E-19
$$

Output:
$$
s_{guess} = 0.76705241700910205677316120102884709893 $$$$
E \approx 1.6591123862520361643 E-21
$$

Input:
$$
r = 0.350000000000000000000000000000000000000000000000000000000000 $$$$
y_0 = 0.310000000000000000000000000000000000000000000000000000000000 ~~~ M_1 = 26 $$$$
\alpha_0 = 2.30000000000000000000000000000000000000000000000000000000000 ~~~ M_2 = 16 $$$$
\rho_0 = 3/4 ~~~ M_3 = 36 $$$$
N_1 = 28 ~~~ N_2 = 32 ~~~ N_3 = 0 $$$$
s_{start} = 0.767052417009102056700000000000000000000000000000000000000000 $$$$
\delta_{start} = 3.00000000000000000000000000000000000000000000000000000000000 E-19
$$

Output:
$$
s_{guess} = 0.76705241700910205677150275933072804301 $$$$
E \approx 6.706881329802452983 E-25
$$

Input:
$$
r = 0.350000000000000000000000000000000000000000000000000000000000 $$$$
y_0 = 0.310000000000000000000000000000000000000000000000000000000000 ~~~ M_1 = 30 $$$$
\alpha_0 = 2.30000000000000000000000000000000000000000000000000000000000 ~~~ M_2 = 20 $$$$
\rho_0 = 3/4 ~~~ M_3 = 40 $$$$
N_1 = 32 ~~~ N_2 = 40 ~~~ N_3 = 0 $$$$
s_{start} = 0.767052417009102056700000000000000000000000000000000000000000 $$$$
\delta_{start} = 3.00000000000000000000000000000000000000000000000000000000000 E-19
$$

Output:
$$
s_{guess} = 0.76705241700910205677150208864814423283 $$$$
E \approx 5.549170067453791727 E-30
$$

Alright, let's see how close we can get to the full number of digits.
\\

\textbf{Note:} I had to increase PARI's stack size to be able to handle the computations below. Each test only took a few minutes, but clearly required a lot of RAM.
\\

Input:
$$
r = 0.350000000000000000000000000000000000000000000000000000000000 $$$$
y_0 = 0.310000000000000000000000000000000000000000000000000000000000 ~~~ M_1 = 60 $$$$
\alpha_0 = 2.30000000000000000000000000000000000000000000000000000000000 ~~~ M_2 = 35 $$$$
\rho_0 = 3/4 ~~~ M_3 = 60 $$$$
N_1 = 62 ~~~ N_2 = 70 ~~~ N_3 = 0 $$$$
s_{start} = 0.767052417009102056771502088642595062500000000000000000000000 $$$$
\delta_{start} = 4.00000000000000000000000000000000000000000000000000000000000 E-37
$$

Output:
$$
s_{guess} = 0.7670524170091020567715020886425950626820135466169027713908 $$$$
E \approx 8.4666453383097228609049572158984559319 E-38
$$

Couldn't do better than 38 digits until I increased the precision again.
\\

\textbf{Precision:} until noted, using default(realprecision, 80) for all tests below.
\\

Input:
$$
r = 0.35000000000000000000000000000000000000000000000000000000000000000000000000000000 $$$$
y_0 = 0.31000000000000000000000000000000000000000000000000000000000000000000000000000000 ~~~ M_1 = 65 $$$$
\alpha_0 = 2.3000000000000000000000000000000000000000000000000000000000000000000000000000000 ~~~ M_2 = 40 $$$$
\rho_0 = 3/4 ~~~ M_3 = 75 $$$$
N_1 = 67 ~~~ N_2 = 80 ~~~ N_3 = 0 $$$$
s_{start} = 0.76705241700910205677150208864259506250000000000000000000000000000000000000000000 $$$$
\delta_{start} = 4.0000000000000000000000000000000000000000000000000000000000000000000000000000000 E-37
$$

Output:
$$
s_{guess} = 0.7670524170091020567715020886425950627666870757682116254668 $$$$
E \approx 7.075768211625466915 E-42
$$

We tied Str\"ombergsson's precision! Now we need to print the secant method steps and see if we think we're doing better.
\\

\textbf{Precision:} went up to default(realprecision, 100) for the computations below.

Input:
$$
r = 0.35 $$$$
y_0 = 0.31 ~~~ M_1 = 75 $$$$
\alpha_0 = 2.3 ~~~ M_2 = 50 $$$$
\rho_0 = 3/4 ~~~ M_3 = 85 $$$$
N_1 = 77 ~~~ N_2 = 100 ~~~ N_3 = 0 $$$$
s_{start} = 0.7670524170091020567715020886425950625 $$$$
\delta_{start} = 4.0 E-37
$$

Output:
$$
s_{guess} = 0.76705241700910205677150208864259506276668707761814379478916307225900858931200 $$$$
E \approx 7.0776181437947891630722590085893120048 E-42
$$

Final set of predictions:
$$
0.76705241700910205677150208864259506276668707761814367412713281055516159638721 $$$$
0.76705241700910205677150208864259506276668707761814375238526869680606787332718 $$$$
0.76705241700910205677150208864259506276668707761814382936130083054086675095072 $$$$
0.76705241700910205677150208864259506276668707761814391545119333396285558223679
$$
This seems to suggest the true eigenvalue lies between
$$
0.7670524170091020567715020886425950627666870776181437 \pm 1e-52
$$
That is, we appear to have found $51$ decimal places.
How can we check this?

\subsection*{Tests Using just Cusp and Flare (Str\"ombergsson's Code)}

Running
\begin{verbatim}
	default(realprecision,80);
	r=0.35; alpha=2.2; y0=0.34; M=60; MM=35;
	zoomin(r,alpha,y0,M,MM,0.7670524170091020567715020886425950625,4e-37);
\end{verbatim}
results in 
\begin{verbatim}
	New predictions:
	7.67052417009102056771502088642595062766687077618142806858366156855094778791353590172130106595438 E-1
	7.67052417009102056771502088642595062766687077618143235168970024034342605904962851352719017652213 E-1
	7.67052417009102056771502088642595062766687077618142234036889656789028573116993374108850338642172 E-1
	7.67052417009102056771502088642595062766687077618141017766554129340601157752762344131337905826968 E-1
	Approximate error:2.217402415894693741448152200507221381111825245393746583451 E-51
\end{verbatim}

So we try starting closer to that value.
Running
\begin{verbatim}
	default(realprecision,80);
	r=0.35; alpha=2.2; y0=0.34; M=80; MM=45;
	zoomin(r,alpha,y0,M,MM,0.767052417009102056771502088642595062766687077618,4e-47);
\end{verbatim}
results in
\begin{verbatim}
	New predictions:
	7.67052417009102056771502088642595062766687077618142701698223967688582627703507519701403781776194 E-1
	7.67052417009102056771502088642595062766687077618142701698223967689282128903537979165535461450436 E-1
	7.67052417009102056771502088642595062766687077618142701698223967661977542295045586530478342260844 E-1
	7.67052417009102056771502088642595062766687077618142701698223967697910545076894113387573752706499 E-1
	Approximate error:3.59330027818485268570954104456546927338 E-65
\end{verbatim}

And more precise:
\begin{verbatim}
	default(realprecision,100);
	r=0.35; alpha=2.2; y0=0.34; M=100; MM=65;
	zoomin(r,alpha,y0,M,MM,0.76705241700910205677150208864259506276668707761814270169822396,4e-62);
\end{verbatim}
results in
\begin{verbatim}
	New predictions:
	7.6705241700910205677150208864259506276668707761814270169822396768661164742932102542... E-1
	7.6705241700910205677150208864259506276668707761814270169822396768661225697612949350... E-1
	7.6705241700910205677150208864259506276668707761814270169822396768661147230540627548... E-1
	7.6705241700910205677150208864259506276668707761814270169822396768661021492553701202... E-1
	Approximate error:2.042050592481475195878038352430303093200589584563292333435 E-69
\end{verbatim}

\section*{March-April 2020}

We want to see how many digits of precision the secant method can achieve in the infinite volume case.
We expect that varying the number of Fourier coefficients used in the two expansions as well as the internal precision PARI will lead to different results.
\\

In both methods below, we write $M_1$ for the number of Fourier coefficients taken in the cuspidal expansion and $M_2$ for the number of coefficients in the flare expansion.
Values in the tables are obtained using the stated method with the parameters
$$
r=\frac{7}{20} ~~~ \alpha=2 ~~~ y_0=0.32 ~~~ s_{\text{start}} = 0.76
$$

\subsection*{Secant Method Results}

We used the parameter
$$
\delta_{\text{start}} = 0.01
$$
for the secant method.

\begin{center}
\begin{tabular}{|c|c|c||c|}
	\hline
	$M_1$ & $M_2$ & PARI precision & Digits correct \\
	\hline \hline
	15 & 5 & 32 &  $\sim8$ \\
	\hline
	15 & 5 & 64 &  $\sim8$ \\
	\hline
	15 & 10 & 32 &  $\sim12$ \\
	\hline
	15 & 15 & 32 &  $\sim12$ \\
	\hline
	15 & 20 & 32 &  $\sim12$ \\
	\hline
	20 & 5 & 32 &  $\sim8$ \\
	\hline
	20 & 10 & 32 &  $\sim16$ \\
	\hline
	20 & 15 & 32 &  $\sim17$ \\
	\hline
	20 & 20 & 32 &  $\sim17$ \\
	\hline
	25 & 5 & 32 &  $\sim9$ \\
	\hline
	25 & 10 & 32 &  $\sim17$ \\
	\hline
	25 & 15 & 32 &  $\sim19$ \\
	\hline
	25 & 15 & 64 &  $\sim19$ \\
	\hline
	25 & 15 & 128 &  $\sim19$ \\
	\hline
	25 & 15 & 256 &  $\sim19$ \\
	\hline
	25 & 15 & 512 &  $\sim19$ \\
	\hline
	25 & 20 & 32 &  $\sim18$ \\
	\hline
	30 & 5 & 32 &  $\sim8$ \\
	\hline
	30 & 10 & 32 &  $\sim16$ \\
	\hline
	30 & 15 & 32 &  $\sim19$ \\
	\hline
	30 & 20 & 32 &  $\sim18$ \\
	\hline
	35 & 15 & 32 &  $\sim19$ \\
	\hline
	40 & 15 & 32 &  $\sim18$ \\
	\hline
	45 & 15 & 32 &  $\sim17$ \\
	\hline
	50 & 15 & 32 &  $\sim18$ \\
	\hline
	50 & 20 & 32 &  $\sim19$ \\
	\hline
	50 & 20 & 64 &  $\sim19$ \\
	\hline
	80 & 50 & 32 &  $\sim2$ \\
	\hline
	80 & 50 & 64 &  $\sim2$ \\
	\hline
\end{tabular}
\end{center}

\subsection*{Grid Method Results}

We used \textbf{FILL IN} grid points for the grid method.

\begin{center}
	\begin{tabular}{|c|c|c||c|}
		\hline
		$M_1$ & $M_2$ & PARI precision & Digits correct \\
		\hline \hline
		25 & 15 & 32 &  $\sim 19$ \\
		\hline
	\end{tabular}
\end{center}
	
\end{document}