\documentclass[]{article}

\usepackage[margin=1.0in]{geometry}
\usepackage{amsmath}
\usepackage{amsfonts}
\usepackage{amsthm}
\usepackage{graphicx}
\usepackage{amssymb}

\usepackage{mathtools}

%opening
\title{Schottky Results}
\author{Alex Karlovitz}
\date{}

\begin{document}
	
	\maketitle

\section*{Known Answers}

McMullen's program ``hdim'' (available on his website) computes the Hausdorff dimension of the limit sets of the symmetric Schottky groups (as well as a variety of other types of ``Kleinian group[s] generated by reflections in the boundaries of a family of \textit{disjoint} round disks in the plane'').
In our notation for Hejhal's algorithm, we deal with the Hausdorff dimension $s$, which is related to the eigenvalue $\lambda$ via the equation
$$
\lambda = s(1 - s)
$$
Here we list some $s$ values for the symmetric Schottky groups with 3 circles of angle $\theta$ for various $\theta$.
\begin{center}
	\begin{tabular}{c|c}
		$\theta$ & $s$ \\
		\hline
		$\pi/6$ & 0.18398306 \\
		$\pi/3$ & 0.29554650 \\
		$\pi/2$ & 0.47218812 \\
		$29\pi/45$ & 0.82342329
	\end{tabular}
\end{center}
\textbf{Note:} in Hejhal's algorithm, we are searching for an $L^2$ function.
A Lax-Phillips argument shows that this can only occur for Hausdorff dimensions greater than $1/2$.
Therefore, we only test on values of $\theta$ which give $s > 1/2$.

\section*{Test with $\theta = 29\pi/45$}

The first test we attempt has parameters
$$
\theta = 29\pi/45 ~~~ N = 100 ~~~ \alpha_0 = 9\pi/10 ~~~ M = 20 ~~~ s_0 = 0.82 ~~~ \delta = 3e-3
$$
(see code for definitions of these parameters).
The test points were randomly selected in the unit circle with argument uniformly sampled from $[0, 2\pi)$ and norm uniformly sampled from $[0, 1)$.
(Points were continually sampled until we found $N$ admissible test points).
Here is the output.
\begin{verbatim}
	--------------------------------------
	Current predictions:
	0.82342993713757314670220170571010269410
	0.82343100886716695319706571452580258151
	
	--------------------------------------
	Current predictions:
	0.82342328823660142656421004716117554899
	0.82342328826886359149131575879717872943
	
	--------------------------------------
	Current predictions:
	0.82342328803459397260593444812684424094
	0.82342328803459397263575544337336559063
	
	--------------------------------------
	Current predictions:
	0.82342328803459397241921860162464331439
	0.82342328803459397241921860162464331581
	
	--------------------------------------
	Current predictions:
	0.8234232880345939723
	0.8234232880345939724
	
	Approximate error: 5.421010862427522170 E-20
	Time: 1024.516 seconds
	Final guess: s = 0.8234232880345939723
\end{verbatim}

The secant method on the above example was run by comparing the second and third Fourier coefficients as predicted on the two sets of test points (the first coefficient is automatically scaled to 1).
After adding the fourth coefficient into secant method computation, this is the output.
\begin{verbatim}
	--------------------------------------
	Current predictions:
	0.82342993713757314670220170571010269410
	0.82343100886716695319706571452580258151
	0.82343163174540417799551991027857319166
	
	--------------------------------------
	Current predictions:
	0.82342328820843524952226842039075081320
	0.82342328823619901530262776313288356772
	0.82342328825229843130676567128115979409
	
	--------------------------------------
	Current predictions:
	0.82342328803459397253876905254446491305
	0.82342328803459397255786284397012620292
	0.82342328803459397256893488636730866952
	
	--------------------------------------
	Current predictions:
	0.82342328803459397241921860162464331450
	0.82342328803459397241921860162464331464
	0.82342328803459397241921860162464331475
	
	--------------------------------------
	Current predictions:
	0.8234232880345939724
	0.8234232880345939724
	0.8234232880345939725
	
	Approximate error: 5.421010862427522170 E-20
	Time: 1379.729 seconds
	Final guess: s = 0.8234232880345939724
\end{verbatim}

I believe that in the previous two examples, the sudden drop in precision on the last step is due to PARI hitting its precision threshold on the secant method step; note that in the penultimate step, the predictions are equivalent almost all the way out to the maximum precision.

In the next test, we use the same parameters as above, except we start at
$$
s_0 = 0.82342328803459397241921860162464331
$$
and we increase PARI's precision with the command \verb|default(realprecision, 60)|.
Here is the output.

\end{document}