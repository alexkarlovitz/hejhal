\documentclass[]{article}

\usepackage[margin=1.0in]{geometry}
\usepackage{amsmath}
\usepackage{amsfonts}
\usepackage{amsthm}
\usepackage{graphicx}
\usepackage{amssymb}

\usepackage{caption}

%opening
\title{Progress on Hejhal's Algorithm}
\author{Alex Karlovitz}
\date{}

\begin{document}
	
	\maketitle
	
\section*{May 3 2019}

I ran Hejhal's algorithm today as described in my Learning Seminar talk (so no special tricks for numerical stability).
I input the following three $r$ values:
\begin{itemize}
	\item $r = 9$
	\item $r = 9.53369526135355727092246524989604949951171875$ (this is approximately the true eigenvalue)
	\item $r = 10$
\end{itemize}
I used parameters
\begin{itemize}
	\item $D = 50$ (digits of precision)
	\item $M_0 = 30$ (this is the value suggested by BSV for $D = 50$)
	\item $Q = 100$ (still not sure how to appropriately choose $Q$)
	\item $Y_1 = 1/2$ and $Y_2 = 1/10$ (these are the two heights at which I take the sample points along a closed horocycle)
\end{itemize}
To check multiplicativity, I printed the Fourier coefficients $a_2, a_3$, and $a_6$ for each $r$.
For $r = 9.533695...$, I got
$$
a_2 = -1.06833 ~~~~~~~~~~ a_3 = -0.45619 ~~~~~~~~~~ a_6 = 0.48737
$$
from height $1/2$ and
$$
a_2 = -1.06833 ~~~~~~~~~~ a_3 = -0.45619 ~~~~~~~~~~ a_6 = 0.48737
$$
from height $1/10$.
Notice that these are exactly the same! The code printed out 50 digits for each of these.
You have to go about 14 digits to start to see a difference.
Moreover, the digits that I have provided exactly match those given by BSV on Str\"ombergsson's website.

I also printed out the errors. That is, for each $r$, I computed the vectors of Fourier coefficients suggested by the sample points at $Y_1$ and $Y_2$.
Then I printed out the 2-norm of the difference between these vectors.
The errors were as follows:
\begin{itemize}
	\item for $r = 9$, error was about $2.8\times 10^{32}$
	\item for $r = 9.533695...$, error was about $4.6\times 10^{19}$
	\item for $r = 10$, error was about $2.06\times 10^{33}$
\end{itemize}
The good news is that the smallest error was given by the $r$ closest to the true eigenvalue.
The bad news is that these errors are huge compared to the size of the Fourier coefficients.

\section*{May 4 2019}

Today, I ran Hejhal's algorithm for real; that is, I let it run the iterative process to zero in on an eigenvalue.
I used the parameters
\begin{itemize}
	\item $D = 50$ (digits of precision)
	\item $M_0 = 30$ (this is the value suggested by BSV for $D = 50$)
	\item $Q = 100$ (still not sure how to appropriately choose $Q$)
	\item $Y_1 = 1/2$ and $Y_2 = 1/10$ (these are the two heights at which I take the sample points along a closed horocycle)
\end{itemize}
and I started with $r$ values $9, 9.1, 9.2, \dots, 9.9, 10$.
After $14$ iterations, the output was
$$
r = 9.53369526135356
$$
This agrees to 14 decimal places with the computations by BSV.
Next, I plan to let the process run a few more iterations.
Judging by yesterday's results, I expect the $r$ value given by my algorithm to start to diverge from the true eigenvalue at around the 15$^{th}$ decimal place.

\section*{May 5 2019}

Starting with
$$
r = 9.53369526135356
$$
which was the result from last time, I ran $8$ more iterations and got
$$
r = 9.5336952613535580388596044418811798095703125
$$
At around the 14$^{th}$ or 15$^{th}$ digit, this starts to diverge from the true eigenvalue.
So it appears that computing everything to $D = 50$ digits of precision allows for $14$ to $15$ digits of precision in the final answer.

The 2$^{nd}$, 3$^{rd}$, and 6$^{th}$ Fourier coefficients given when $r$ is input the 14 decimal places of accuracy are as follows:
$$
a_2 = -1.0683335512232188680742305108912588987488213870376 $$$$
a_3 = -0.45619735450604610015637972856710657995012991007215 $$$$
a_6 = 0.48737093979348801248710895531524614714300871266411
$$
These agree with the values on Str\"ombergsson's website to 12 digits, 12 digits, and 10 digits, respectively.

\section*{June 25 2019}

I repeated the test with the same inputs as above except with $Y_1 = 1/300$ and $Y_2 = 1/50,000$.
This immediately went awry, choosing $r = 10.3$ at the first step instead of the true $r = 9.5$.
This surprised me, since lower values of $Y_1$ and $Y_2$ will cause the sample points to hit many more fundamental domains.
I expected this to cause the linear system to have more information, and hence cause the algorithm to converge more quickly.
\\

I have three guesses for why this test didn't work as expected:
\begin{itemize}
	\item I didn't use mpmath right away when I initialized $Y_1$ and $Y_2$; perhaps this caused more rounding error because of the small numbers.
	\item Maybe $1/300$ and $1/50,000$ are too close together for the linear system to have a lot of information.
	\item Perhaps there is an error in my above reasoning.
\end{itemize}

\section*{June 26 2019}

Regarding the guesses listed above, I was able to quickly rule out the first guess.
Next, to think about the second guess, I input the heights $Y_1 = 1/2$ and $Y_2 = 1/50,000$, expecting this to do better than any previous experiment.
The output was
$$
r = 9.533695261353561603636144050396978855133056640625
$$
This is correct to about the 13$^{th}$ or 14$^{th}$ decimal place, which is one digit worse than the original $Y_1 = 1/2$ and $Y_2 = 1/10$ case.

\section*{July 7 2019}

I ran Hejhal's algorithm today for a new group; namely, the group
\[
	\Gamma_4 = \left\langle 
	\begin{pmatrix}
		1 & \sqrt{2} \\
		0 & 1
	\end{pmatrix},
	\begin{pmatrix}
		0 & -1 \\
		1 & 0
	\end{pmatrix}	
	 \right\rangle
\]
To make Fourier expansion at the cusp the same is in the $SL(2, \mathbb{Z})$ case, I in fact used a group conjugate to $\Gamma_4$:
\[
	\left\langle 
	\begin{pmatrix}
		1 & 1 \\
		0 & 1
	\end{pmatrix},
	\begin{pmatrix}
		0 & -\frac{1}{\sqrt{2}} \\
		\sqrt{2} & 0
	\end{pmatrix}	
	\right\rangle
\]
According to a paper by Hejhal (``On Eigenvalues of the Laplacian for Hecke Triangle Groups''), $\Gamma_4$ has $\lambda = 1/4 + r^2$ as an eigenvalue where $r = 11.317680\dots$.

I ran the algorithm with parameters
\begin{itemize}
	\item $D = 50$
	\item $M_0 = 30$
	\item $Q = 100$
	\item $Y_1 = 1/2$ and $Y_2 = 1/10$
\end{itemize}
This did not work.
Next, I tried running the code to a higher precision and with more sample points (along with different choices for $Y_1$ and $Y_2$).
Specifically, I used the parameters
\begin{itemize}
	\item $D = 80$
	\item $M_0 = 80$
	\item $Q = 100$
	\item $Y_1 = 1/3$ and $Y_2 = 1/20$
\end{itemize}
This \textit{did} work!
It found $r = 11.317680$ to those $6$ decimal places, which is the same precision to which Hejhal found the eigenvalue.
\\

I plan to mess around with the precision and number of sample points to see how high I need to go to make the code work.

\section*{July 8 2019}

I reran yesterday's test with the parameters
\begin{itemize}
	\item $D = 50$
	\item $M_0 = 50$
	\item $Q = 100$
	\item $Y_1 = 1/3$ and $Y_2 = 1/20$
\end{itemize}
This worked!

This time, I was looking for the eigenvalue $\lambda = 1/4 + r^2$ where $r = 7.220872\dots$.
Again, the code found this eigenvalue to $6$ places of accuracy.

\section*{July 16 2019}

I realized that I am unsure why the number of test points must be larger than the size of the final system.
In fact, I reran the tests from July 8 with the following parameters
\begin{itemize}
	\item $D = 50$
	\item $M_0 = 50$
	\item $Q = 10$
	\item $Y_1 = 1/3$ and $Y_2 = 1/20$
\end{itemize}
This worked to a reasonable degree!

In six steps, it found the eigenvalue $\lambda = 1/4 + r^2$ where $r = 7.220872\dots$ to 4 decimal places.
Of course, using less points gives less information.
So it is not surprising that this worked to less decimal places.
	
\end{document}