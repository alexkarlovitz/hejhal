\documentclass[]{article}

\usepackage[margin=1.0in]{geometry}
\usepackage{amsmath}
\usepackage{amsfonts}
\usepackage{amsthm}
\usepackage{graphicx}
\usepackage{amssymb}

\usepackage{mathtools}

\usepackage[
backend=bibtex,
style=alphabetic,
sorting=ynt
]{biblatex}
\addbibresource{refs}

\usepackage{hyperref}
\PassOptionsToPackage{hypernotes=false}{hyperref}

%opening
\title{Dirichlet Domains}
\author{Alex Karlovitz}
\date{}

\begin{document}
	
	\maketitle
	
In this note, we discuss the details involved in computing a Dirichlet domain for a group $\Gamma$ acting discretely on the upper half-plane model of hyperbolic 2-space.
As an example, we construct the compact arithmetic surface described by Kontorovich in \cite{kontorovich2011}.
	
\section*{Constructing a Dirichlet Domain}
	
To construct a Dirichlet domain, one starts with a specific point $z \in \mathbb{H}$ (we often take $z = 2i$).
Then, we find points in the orbit of $z$ under $\Gamma$ which are relatively close to $z$.
For each such point $z_*$, we find the geodesic of points which are equidistant from $z$ and $z^*$.
The set of points in $\mathbb{H}$ which are closer to $z$ than to any of the $z_*$'s will constitute a fundamental domain for $\Gamma\backslash\mathbb{H}$ so long as we took enough points in the orbit.

Here is pseudocode for finding the geodesic of points equidistant from two points $z_1$ and $z_2$ in $\mathbb{H}$.
\begin{verbatim}
def find_equidistant_geodesic(z1, z2) :
    # give names to real and imaginary parts for ease of reading
    x1, y1 = real(z1), imag(z1)
    x2, y2 = real(z2), imag(z2)

    # Step 1: get geodesic G through z1 and z2
    
    if x1 == x2 :
        # G is vertical line
        G = line((x1, 0), (x1, y1))
    else :
        # G is circle of center c and radius r
        c = (|z2|^2 - |z1|^2)/2/(x2 - x1)
        r = sqrt( (x1 - c)^2 + y1^2 )
        G = circle(c, r)
        
    # Step 2: get point z0 on G equidistant from z1 and z2 (in hyperbolic distance)
    
    eqn1 = |z - z1|^2/4/imag(z)/y1 == |z - z2|^2/4/imag(z)/y2
    if isCircle(G) :
        eqn2 = |z - c|^2 == r^2
    else :
        eqn2 = real(z) == x1
    z0 = solve((eqn1, eqn2), unknown=z)
    
    # Step 3: get geodesic G0 through z0 which meets G at a right angle
    
    x0, y0 = real(z0), imag(z0)
    if isCircle(G) :
        # first get slope of G at z0
        s = (c - x0)/y0
        
        # if the slope is 0, return vertical line through z0
        if s == 0 :
            return line((x0, 0), (x0, y0))
        
        # solve for center a and radius R
        a = -y0/s + x0
        R = sqrt( (x0 - a)^2 + y0^2 )
        
        G0 = circle(a, R)
    else :
        # z0 is at top of circle
        G0 = circle(x0, y0)
        
    return G0
\end{verbatim}
	
	\pagebreak
	
	\printbibliography
	
\end{document}