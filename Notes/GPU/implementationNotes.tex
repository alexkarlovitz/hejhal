\documentclass[]{article}

\usepackage[margin=1.0in]{geometry}
\usepackage{amsmath}
\usepackage{amsfonts}
\usepackage{amsthm}
\usepackage{graphicx}
\usepackage{amssymb}

\usepackage{mathtools}

\newtheorem*{theorem}{Theorem}

%opening
\title{GPU-Acceleration of Hejhal's Algorithm}
\author{Alex Karlovitz}
\date{}

\begin{document}
	
	\maketitle
	
A \textit{graphics processing unit} (GPU) is a specialized electronic circuit originally designed for efficient computer graphics and image processing.
GPUs support parallel computation on large blocks of data, making them useful for large-scale linear algebra routines.
In this document, we go through the details for running Hejhal's algorithm on a GPU.
The software is implemented in PyTorch, a python library for machine learning which supports GPU operations.

\section*{Appendix: Series Expansion of the $K$-Bessel Function}

The $K$-Bessel function is a solution to the modified Bessel's equation
\begin{equation}\label{modifiedBessel}
	x^2y'' + xy' - (x^2 + \nu^2)y = 0
\end{equation}
where $\nu$ is a (complex) parameter.
Specifically, one defines a pair of standard solutions $I_\nu$ and $K_\nu$ to (\ref{modifiedBessel}) as follows.
$$
I_\nu(x) = \left(\frac{x}{2}\right)^\nu \sum_{k = 0}^{\infty}\frac{1}{k!(\nu + 1)_k}\left(\frac{x}{2}\right)^{2k}
$$
where $(a)_n$ is the rising Pochhammer symbol.
$K_\nu$ is the solution to (\ref{modifiedBessel}) with the asymptotic
$$
K_\nu(x) \sim \sqrt{\pi/(2x)}e^{-x}
$$
as $x \rightarrow \infty$.
To write down a series expansion, we can use the formula
$$
K_\nu(z) = \frac{1}{2}\pi\frac{I_{-\nu}(z) - I_\nu(z)}{\sin(\nu\pi)}
$$
Note: this formula is valid only for non-integer $\nu$; however, the limit exists as $\nu$ approaches integers, so we can extend the formula in this manner.

In any case, $\nu$ will not be an integer in the applications to Hejhal's algorithm, so we can directly use the relation to $I_\nu$ to derive a series expansion for $K_\nu$.
Combining the series expansions for $I_{-\nu}$ and $I_\nu$, this gives
$$
K_\nu(z) = \frac{\pi}{2\sin(\nu\pi)} \sum_{k = 0}^{\infty} \frac{(z/2)^{2k}}{k!}\left( \frac{(z/2)^{-\nu}}{(1 - \nu)_k} - \frac{(z/2)^\nu}{(1 + \nu)_k} \right)
$$
	
\end{document}